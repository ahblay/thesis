\chapter{Thickness One}

While results from the previous chapters resolve the question of $m(a_1,a_2,a_3,3)$ for $a_1, a_2 , a_3 \geq 11$, our constructions for smaller grids remain incomplete. Nevertheless, computer examples seem to suggest that most grids of minimum size at least 2 are optimal. Grids of thickness 1 tell a different story. In this chapter, we prove that the only perfect grids in thickness 1 are those of the form $[2^n-1]^2$. This answers a question posed by Benevides, Bermond, Lesfari and Nisse in \cite{benevides:2021}.

The broad structure of the proof is as follows: Let $A_0$ be a perfect lethal set on the grid $G(a_1, a_2, 1)$. We show that the structure of $A_0$ guarantees both that $a_1,a_2$ are odd, and that there exists a perfect lethal set on the smaller grid $G(\frac{a_1-1}{2}, \frac{a_2-1}{2}, 1)$. Repeated applications of this process of reduction guarantee the existence of a perfect lethal set on the grid $G(a_0, 1,1)$. Since the only such grid that admits a perfect lethal set is $G(1,1,1)$, we are forced to conclude that $a_1 = a_2 = 2^k-1$ for some $k > 0$. 

\section{Preliminaries}
For the remainder of the chapter, let $G = [a_1] \times [a_2]$. Recall that every perfect lethal set matches the surface area bound. In particular,
$$|A_0| = \frac{a_1a_2 + a_1 + a_2}{3}.$$
We begin with the following observations regarding the structure of $A_0$:

\begin{prop}
\label{prop:alternating_border}
If $A_0$ is a perfect lethal set on $G$, then $A_0$ contains alternating vertices along the border of $G$. 
\end{prop}

\begin{proof}
Since $A_0$ is perfect, it must form an independent set in $G$. By Proposition \ref{prop:border}, no two adjacent border vertices are both uninfected. Together, these conditions ensure that $A_0$ intersects the border of $G$ in an alternating pattern (see Figure \ref{fig:border}). 
\end{proof}

\begin{figure}[]
\centering
\includegraphics[width=0.5\textwidth]{figures/6/border.pdf}
\caption{Alternating infection along the border of $[7] \times [13]$.}
\label{fig:border}
\end{figure} 

\begin{prop}
\label{prop:odd_by_odd}
If $A_0$ is a perfect lethal set on $G$, then $a_1, a_2 \equiv 1 \pmod 2$.
\end{prop}

\begin{proof}
By Propositions \ref{prop:alternating_border} and \ref{prop:corners}, $a_1, a_2 \equiv 1 \pmod 2$.
\end{proof}

\begin{prop}
\label{prop:one_border_vertex}
Let $A_0$ be a perfect lethal set on $G$ under $3$-neighbor percolation. Let $H$ be the subgraph of $G$ induced by $V(G) \setminus A_0$. Then $H$ is acyclic and each component of $H$ contains exactly one border vertex.
\end{prop}

\begin{proof}
By Proposition \ref{prop:immune_regions}, we have that each component of $H$ contains at most one border vertex. We show that each component of $H$ contains exactly one border vertex.

By assumption, $H$ is a forest. Therefore, the number of components of $H$ is given by $|V(H)| - |E(H)|$. Note that $|V(H)| = |V(G)| - |A_0|$. Since $A_0$ is perfect and lethal,
\begin{align*}
|V(H)| &= a_1a_2 - \frac{1}{3}(a_1a_2+a_1+a_2) \\
&= \frac{1}{3}(2a_1a_2-a_1-a_2).
\end{align*}
To determine $|E(H)|$, we calculate the number of edges removed from $G$ to obtain $H$. Recall that $A_0$ is an independent set. Therefore, no two vertices of $A_0$ remove the same edge. Since all interior vertices of $G$ have degree 4, we have that the number of edges removed from $G$ to obtain $H$ is
$$\frac{4}{3} \left(a_1a_2+a_1+a_2 \right) - (a_1+a_2-2) - 4,$$
where the terms $(a_1+a_2-2)$ and $4$ account for the border vertices of degree $3$ and degree $2$, respectively. Therefore, the number of edges in $H$ is
\begin{align*}
|E(H)| &= |E(G)| - \frac{4}{3} \left(a_1a_2+a_1+a_2 \right) - (a_1+a_2-2) - 4 \\
&= 2a_1a_2 -a_1 -a_2 - \frac{1}{3}(4a_1a_2+a_1+a_2-6) \\
&= \frac{1}{3}(2a_1a_2-4a_1-4a_2+6).
\end{align*}
The number of components of $H$ is given by
\begin{align*}
|V(H)| -|E(H)| &= \frac{1}{3}(2a_1a_2-a_1-a_2) - \frac{1}{3}(2a_1a_2-4a_1-4a_2+6) \\
&= a_1 + a_2 -2.
\end{align*}
As there are exactly $a_1 + a_2 - 2$ border vertices in $H$, each component must contain exactly one border vertex.
\end{proof}

Consider a labeling of the vertices of $G$ by their coordinates, starting at $(1,1)$ in the lower left and ranging to $(a_1,a_2)$ in the upper right. Refer to a vertex $(x,y)$ as ``even" or ``odd" depending on the parity of $x+y$. If a set $S \subseteq V(G)$ contains all vertices of the same parity, call $S$ monochromatic. The following lemma leverages the prior propositions to prove that any perfect lethal set on $G$ must be monochromatic.

\begin{lem}
Let $A_0$ be a perfect lethal set on $G$. Then $A_0$ is monochromatic with respect to the proper $2$-coloring of $G$.
\end{lem}

\begin{proof}
From Proposition \ref{prop:alternating_border}, observe that $A_0$ contains all even vertices along the border of $G$. Suppose for contradiction that $A_0$ also contains odd vertices. We show that this implies the existence of a cycle in the subgraph induced by $V(G) \setminus A_0$, contradicting Proposition \ref{prop:one_border_vertex}. 

Let $H$ be a graph with vertices $V(H) = V(G)$ and edges $uv$ if and only if $u$ and $v$ are diagonally adjacent in $G$. Consider the subgraph of $H$ induced by the odd vertices of $A_0$ and let $K$ be a connected component. Observe that $K$ is acyclic: any cycle in $K$ encloses a component of $G[\overline{A_0}]$, contradicting Proposition \ref{prop:one_border_vertex}. Furthermore, by Proposition \ref{prop:alternating_border}, all vertices of $K$ are in the interior of $G$. Let $C_H$ be the cycle induced in $H$ by $N_G(K)$. Note that since $A_0$ is an independent set, $N_G(K) \cap A_0 = \emptyset$ and $C_H \cap A_0 = \emptyset$. Consider the closed walk induced in $G$ by the vertices $V(C_H) \cup N_H(K) \setminus A_0$. This walk describes a cycle $C_G$ in $G[\overline{A_0}]$, which contradicts Proposition \ref{prop:one_border_vertex}.
 \end{proof}

\begin{figure}[]
\centering
\includegraphics[width=0.5\textwidth]{figures/6/monochromatic.pdf}
\caption{$[7] \times [13]$ grid with component $K$ (red), $C_H$ (blue), and $C_G$ (dashed).}
\label{fig:border}
\end{figure} 

\section{Reduction}
Using propositions from the prior section, we show that it is possible to obtain a perfect lethal set on the smaller grid $G' = G(\frac{a_1-1}{2}, \frac{a_2-1}{2}, 1)$ obtained from $G$. Let the vertices of $G'$ be $2 \times 2$ tiles of $G$ given by
$$\{(2x-1,2y-1),(2x-1,2y),(2x,2y-1),(2x,2y) \mid (x,y) \in [1, \tfrac{a_1-1}{2}] \times [1, \tfrac{a_2-1}{2}]\},$$
with adjacencies between tiles that differ by one in each of the cardinal directions. Note that Proposition \ref{prop:odd_by_odd} ensures that $|V(G')|$ is an integer. Furthermore, observe that for any tile $T_{x,y} \in V(G')$, $|A_0 \cap T_{x,y}| \in \{1,2\}$. This follows from the fact that $A_0$ is an independent set, and $G[\overline{A_0}]$ is acyclic. For all $T_{x,y} \in V(G')$, color $T_{x,y}$ blue if $|A_0 \cap T_{x,y}| = 2$, and white otherwise. Let $b$ and $w$ be the number of blue and white tiles in $V(G')$, respectively. We determine $b$ by solving the following system of equations:
\begin{align*}
b + w &= \frac{(a_1-1)(a_2-1)}{4} \\
2b + w &= \frac{a_1a_2+a_1+a_2}{3} - \frac{a_1+a_2}{2}.
\end{align*}
This gives the following expression for $b$:
\begin{align}
\frac{a_1a_2+a_1+a_2}{3} - \frac{a_1+a_2}{2} - \frac{(a_1-1)(a_2-1)}{4} &= \frac{a_1a_2+a_1+a_2-3}{12} \label{eq:tile_bound_1} \\
&= \frac{(\frac{a_1-1}{2})(\frac{a_2-1}{2}) + \frac{a_1-1}{2} + \frac{a_2-1}{2}}{3} \label{eq:tile_bound_2}.
\end{align}
Note that this is precisely the surface area bound for the $(\frac{a_1-1}{2}, \frac{a_2-1}{2}, 1)$ grid. Furthermore, since $a_1 \equiv a_2 \equiv 1 \pmod 6$ or $a_1 \equiv a_2 \equiv 3 \pmod 6$, all terms on the LHS of Equation \ref{eq:tile_bound_1} are integral, and so the SA bound in Equation \ref{eq:tile_bound_2} is tight.

\begin{figure}[]
\centering
\includegraphics[width=0.5\textwidth]{figures/6/tiles.pdf}
\caption{$[7] \times [13]$ grid with $T_{x,y}$ colored blue if $|T_{x,y} \cap A_0| = 2$. Note that $A_0$ is \emph{not} perfect.}
\label{fig:tiles}
\end{figure} 

We prove that the blue tiles form a lethal set in $G'$. We begin with the following observation:

\begin{prop}
\label{prop:bottom_left}
All white tiles have their $A_0$-vertex in the bottom left corner.
\end{prop}

\begin{proof}
For contradiction, suppose that there exists a white tile $T_0$ with one infected vertex in the upper right. By Proposition \ref{prop:one_border_vertex}, there exists a path in $G[\overline{A_0}]$ from $T_0 \setminus A_0$ to the border. We consider the sequence of white tiles $T_0, \dots, T_n$ containing this path. 

%Consider two consecutive tiles $T_i, T_{i+1}$ in this sequence. Note that $T_i$ and $T_{i+1}$ cannot be diagonally adjacent, as such a configuration creates a 4-cycle in $G[\overline{A_0}]$ (see Figure \ref{fig:tile_cycle}). 

Observe that, by Proposition \ref{prop:alternating_border}, $T_n$ has its infected vertex in the bottom left corner. Therefore, since $T_0$ contains an infection in the top right by assumption, there exist consecutive tiles $T_i, T_{i+1}$ such that $T_i$ has an infection in the top right, and $T_{i+1}$ has an infection in the bottom left. Note that $T_i$ and $T_{i+1}$ cannot be diagonally adjacent, as such a configuration creates a 4-cycle in $G[\overline{A_0}]$ (see Figure \ref{fig:tile_cycle}).

We consider two cases. If $T_{i+1}$ is below or to the left of $T_i$, we obtain a 4-cycle. On the other hand, if $T_{i+1}$ is above or to the right of $T_i$, there is no path in $G[\overline{A_0}]$ between them (see Figure \ref{fig:tile_cases}). We therefore conclude that $T_0$ must have an infected vertex in the bottom left.
\end{proof}

\begin{figure}[]
\centering
\includegraphics[width=0.15\textwidth]{figures/6/tile_cycle.pdf}
\caption{Diagonal white tiles and the resulting 4-cycle.}
\label{fig:tile_cycle}
\end{figure} 

\begin{figure}[]
\centering
\includegraphics[width=0.5\textwidth]{figures/6/tile_cases.pdf}
\caption{Possible configurations of adjacent white tiles.}
\label{fig:tile_cases}
\end{figure} 

We are now prepared to prove that the blue tiles form a lethal set in $G'$.

\begin{lem}
\label{lem:blue_lethal}
The set of blue tiles is lethal and perfect in $[\tfrac{a_1-1}{2}] \times [\tfrac{a_2-1}{2}]$ under 3-neighbor percolation.
\end{lem}

\begin{proof}
In Equation \ref{eq:tile_bound_2}, we saw that the number of blue tiles matches the lower bound for 3-neighbor percolation in $[(a_1-1)/2] \times [(a_2-1)/2]$. We now show that the 3-neighbor process infects white tiles if and only if they are adjacent to at least 3 blue tiles.

For sufficiency, consider the four cases illustrated in Figure \ref{fig:tile_infection}. In each of these configurations, the upper right vertex of the white tile (labeled with a ``2") becomes infected after two iterations. Each case requires the assistance of one to two extra infections outside of the three blue tiles. However, these infections constitute the bottom left vertex in adjoining tiles, which is always infected.

For necessity, we show that any cycle or border-to-border path in the white tiles of $G'$ implies a cycle or border-to-border path in $G[\overline{A_0}]$. Observe that, by Proposition \ref{prop:bottom_left}, the vertices $(T_i \cup T_{j}) \setminus A_0$ of any adjacent white tiles $T_i, T_j$ induce a connected component in $G$. Therefore, any cycle or border-to-border path in $G'$ implies the existence of a cycle or border-to-border path in $G[\overline{A_0}]$. We conclude that the blue tiles form a perfect lethal set in $[(a_1-1)/2] \times [(a_2-1)/2]$ under 3-neighbor percolation.
\end{proof}

\begin{figure}[]
\centering
\includegraphics[width=\textwidth]{figures/6/tile_infection.pdf}
\caption{The four configurations of blue tiles leading to infection.}
\label{fig:tile_infection}
\end{figure} 

%Proposition \ref{prop:odd_by_odd} mandates that all perfect lethal sets on $[a_1] \times [a_2]$ be odd-by-odd.

We have shown that the existence of a perfect lethal set on $[a_1] \times [a_2]$ implies the existence of a perfect lethal set on $[(a_1-1)/2] \times [(a_2-1)/2)]$. Suppose, without loss of generality, that $a_1 \geq a_2$. Then, by Proposition \ref{prop:odd_by_odd}, it must be the case that $a_2 = 2^{k}-1$ for some $k> 0$. By repeated applications of Lemma \ref{lem:blue_lethal}, we ultimately obtain a grid $[a_0] \times [1]$ that admits a perfect lethal set. Clearly, the only such grid is the single vertex. Therefore, it must follow that $a_1 = a_2 = 2^k-1$. We conclude that the only two-dimensional grids that admit perfect lethal sets under 3-neighbor percolation are square grids of the form $[2^n-1]^2$. Furthermore, this process fixes the location of infected vertices in a particular configuration. This configuration is published in Benevides et al \cite{benevides:2021}, and reproduced below.

\section{Purina}

We refer to this construction colloquially as the Purina construction, due to the similarly between its instance $G(3,3,1)$ and the logo of the pet food brand. No funding has been offered, but we are open to the possibility. A more extensive discussion on this pattern can be found in \cite{benevides:2021}.

\begin{con}[Benevides, Bermond, Lesfari, Nisse]
\label{con:purina}
All grids of the form $G(2^n-1, 2^n-1, 1)$ are perfect.
\end{con}

\begin{figure}[]
\centering
\includegraphics[width=0.1\textwidth]{figures/7/3x3x1.pdf}
\caption{A perfect percolating set for $G(3,3,1)$.}
\label{fig:3x3x1}
\end{figure} 

\begin{proof}
This is a recursive construction built from the base component piece shown in Figure \ref{fig:3x3x1}. Note that this $G(3,3,1)$ construction is lethal under the 3-neighbor bootstrap process, and that it meets the surface area bound:
$$\frac{1}{3} (ab+bc+ca) = \frac{1}{3} (9 + 3 + 3) = 5.$$
For larger grids of size $G(2^n-1, 2^n-1, 1)$, join four copies of $G(2^{n-1}-1, 2^{n-1}, 1)$ about two perpendicular corridors, and infect the vertex at their intersection (Figure \ref{fig:15x15x1}). Observe that the resulting set is lethal: each of the four smaller grids is lethal by hypothesis, and the remaining vertices induce a forest with disconnected boundary points, which percolates by Proposition \ref{prop:immune_regions}. Furthermore, note that
\begin{align*}
SA(2^n-1,2^n-1,1) &= \frac{1}{3} (2^{2n}-1) \\
&= 4 \cdot \frac{1}{3} (2^{2n-2} -1) + 1 = 4 \cdot SA(2^{n-1}-1, 2^{n-1}, 1) + 1,
\end{align*}
and therefore this construction is perfect.
\end{proof}

\begin{figure}[]
\centering
\includegraphics[width=0.6\textwidth]{figures/7/15x15x1.pdf}
\caption{A perfect percolating set for $G(15,15,1)$.}
\label{fig:15x15x1}
\end{figure} 

Although the only two-dimensional grids that admit perfect lethal sets under 3-neighbor percolation are square grids of the form $[2^n-1]^2$, there is at least one family of two-dimensional grids that admits optimal lethal sets (i.e. meets the rounded surface area bound). These sets are qualitatively different from the Purina construction: while Purina exhibits an almost recursive structure, the sets presented in Construction \ref{con:snake} rely on ``corridors," down which the infection spreads. The notion of ``corridors" is not new; constructions with much of the same character as these are given for non-Purina grids in \cite{benevides:2021}. We examine such grids in the following chapter, making particular note of their use in our analysis of other infinite families of grids that admit perfect lethal sets. 

