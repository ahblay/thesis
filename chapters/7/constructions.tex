\chapter{Constructions}

\section{Introduction}

In this chapter, we present diagrammed proofs of lethal sets that percolate at the lower bound. The proofs are organized by the thickness of the grid. Many of the constructions in the following sections belong to infinite families of either optimal or perfect sets. In this case, we shall examine the grids by region, and observe that certain regions can be expanded to arbitrarily large sizes using mathematical induction. In all constructions, we consider percolation under the 3-neighbor process.

We shall call a thickness \emph{semi-complete} if all divisibility cases are optimal. We present some useful definitions and lemmas.

\begin{defn}
For a grid $G=[a_1] \times [a_2] \times [a_3]$, define the $k$th \emph{level} of $G$ as the subgraph $L_k = [a_1] \times [a_2] \times \{k\}$, for $k \in [a_3]$. 
\end{defn}

\begin{lem}
\label{lem:2_neighbor_levels}
Let $G=[a_1] \times [a_2] \times [a_3]$ and let $L_k$ be the $k$th level of $G$. Suppose all vertices in $L_k$ are infected. Then any lethal set in $L_{k+1}$ (resp. $L_{k-1}$) under the 2-neighbor process is lethal in $G[V(L_{k}) \cup V(L_{k+1})]$ (resp. $G[V(L_{k-1}) \cup V(L_{k})]$) under the 3-neighbor process. 
\end{lem}

\begin{proof}
Each vertex $v \in L_{k+1} \cup L_{k-1}$ has an infected neighbor in $L_k$. Therefore, if $v$ has two infected neighbors in its own level, it has at least 3 infected neighbors in $G$. 
\end{proof}

\section{Thickness 1}

There are two general constructions in thickness 1 that percolate at the surface area bound. The first construction is perfect for all $(2^n-1, 2^n-1, 1)$ grids, and originates in a 2021 paper by Benevides et al. \cite{benevides:2021}. The second construction is optimal for all grids $(a,b,1)$, where $a \equiv 5 \pmod 6$, $b \equiv 1 \pmod 2$, and $a,b \geq 5$. As such grids constitute non-divisibility cases, this construction is not perfect.

% Purina constructions that percolate at the S.A. bound
\subsection{Purina}

We refer to this construction colloquially as the Purina construction, due to the similarly between its instance on the $(3,3,1)$ grid and the logo of the pet food brand. No funding has been offered, but we are open to the possibility. A more extensive discussion on this pattern can be found in \cite{benevides:2021}.

\begin{con}
\label{con:purina}
All grids of the form $(2^n-1, 2^n-1, 1)$ are perfect.
\end{con}

\begin{figure}[]
\centering
\includegraphics[width=0.1\textwidth]{figures/7/3x3x1.pdf}
\caption{A perfect percolating set for $(3,3,1)$.}
\label{fig:3x3x1}
\end{figure} 

\begin{proof}
This is a recursive construction built from the base component piece shown in figure \ref{fig:3x3x1}. Note that this $(3,3,1)$ construction is lethal under the 3-neighbor bootstrap process, and that it meets the surface area bound:
$$\frac{1}{3} \cdot (ab+bc+ca) = \frac{1}{3} \cdot (9 + 3 + 3) = 5.$$
For larger grids of size $(2^n-1, 2^n-1, 1)$, join four copies of $(2^{n-1}-1, 2^{n-1}, 1)$ about two perpendicular corridors, and infect the vertex at their intersection (figure \ref{fig:15x15x1}). Observe that the resulting set is lethal: each of the four smaller grids is lethal by hypothesis, and the remaining vertices induce a forest with disconnected boundary points, which percolates by lemma \ref{lem:forest}. Furthermore, note that
\begin{align*}
\text{S.A.}(2^n-1,2^n-1,1) &= \frac{1}{3} \cdot (2^{2n}-1) \\
&= 4 \cdot \frac{1}{3} \cdot (2^{2n-2} -1) + 1 = 4 \cdot \text{S.A.}(2^{n-1}-1, 2^{n-1}, 1) + 1,
\end{align*}
and therefore this construction is perfect.
\end{proof}

\begin{figure}[]
\centering
\includegraphics[width=0.6\textwidth]{figures/7/15x15x1.pdf}
\caption{A perfect percolating set for $(15,15,1)$.}
\label{fig:15x15x1}
\end{figure} 

% (odd)x(odd) constructions that percolate at the S.A. bound
\subsection{Snakes}

As indicated by lemma \ref{lem:forest}, a fundamental characteristic of lethal sets $S$ is the presence of an initially uninfected corridor, bounded by walls of infection. This structure is apparent in the second diagrams of figures \ref{fig:15x15x1} and \ref{fig:11x13x1}. These corridors correspond to forests in the complement $G[\overline{S}]$ of $S$. In this subsection, we provide a general method for constructing such corridors in $(a, b, 1)$ grids where $a \equiv 5 \pmod 6$ and $b \equiv 1 \pmod 2$.

% WITH THE EXCEPTION OF WIDTH 3!!!!!
% WIDTH >= 5
\begin{con}
\label{con:snake}
All grids of the form $(a,b,1)$, $a \equiv 5 \pmod 6$, $b \equiv 1 \pmod 2$, and $a,b \geq 5$ are optimal.
\end{con}

\begin{proof}
For grids of the form $(a,b,1)$, $a \equiv 5 \pmod 6$, $b \equiv 1 \pmod 2$, we construct an optimal infected set and show that it percolates by lemma \ref{lem:forest}. For the base case, consider the $(5,5,1)$ grid $G$ illustrated in figure \ref{fig:5x5x1}. Observe that this construction is optimal. Now consider the grid $G'$ resulting from the insertion of a $(5, 2k, 1)$ block, as shown in figure \ref{fig:5x13x1}. Note that the subgraph induced by the uninfected vertices of $G'$ satisfies the conditions of lemma \ref{lem:forest}. Furthermore, note that if any $(5, n, 1)$ grid is optimal, the $(5,n+2,1)$ grid resulting from such a construction has surface area bound $\text{S.A.}(5,n,1) + 4$, which agrees with the number of infected vertices.

\begin{figure}[]
\centering
\includegraphics[width=0.15\textwidth]{figures/7/5x5x1.pdf}
\caption{An optimal percolating set for $(5,5,1)$.}
\label{fig:5x5x1}
\end{figure} 

\begin{figure}[]
\centering
\includegraphics[width=0.5\textwidth]{figures/7/5x13x1.pdf}
\caption{An optimal percolating set for $(5,13,1)$.}
\label{fig:5x13x1}
\end{figure} 

To extend this construction in the vertical direction, we introduce a kink in the snaking infection. This kink requires six rows to produce a repeating pattern. The structure of this design is shown in figure \ref{fig:11x13x1}. For grids of smaller width, the same construction gives optimal percolating sets; however, the snaking pattern is increasingly difficult to recognize in thin grids.
\end{proof}

\begin{figure}[]
\centering
\includegraphics[width=0.6\textwidth]{figures/7/11x13x1.pdf}
\caption{An optimal percolating set for $(11,13,1)$.}
\label{fig:11x13x1}
\end{figure} 

% Maybe other (divisibility) constructions that percolate at 1 over the bound (if they exist)?

\section{Thickness 2}

In this section we examine four infinite families of perfect grids. We show that each has a manifold that admits a lethal set of perfect size. We note that such lethal sets are likely to exist for nearly all divisibility cases in thickness two; however, constructions are elusive and those presented here are sufficient to prove the main result of this thesis.

Unlike those presented in the previous section, the following proofs all leverage lemma \ref{lem:walls}. As a consequence, their argumentative structure remains broadly the same, even as the constructions themselves appear quite different. For this reason, we shall outline this structure here, before examining the specific proofs. 

We begin by demonstrating that a grid $G$ admits a manifold $M$. To show so, we identify the regions $R_1, \dots, R_k$ that partition $V(G) \setminus M$. In our diagrams, these regions are represented by the volumes enclosed by three perpendicular blue, green, and red walls. We then identify a proper unfolding $H$ of $M$ and show that $H$ admits a lethal set $A$, where $|A| = \text{SA}(G)$. Finally, we apply corollary \ref{cor:ortho_walls} to prove that $G$ is perfect. 

\begin{con}
All $(a,3,2)$ grids with $a \equiv 0 \pmod 6$ are perfect. 
\end{con}

\begin{proof}
Let $G$ be an $(a,3,2)$ grid with $a \equiv 0 \pmod 6$, and let $M$ be a manifold of $G$ and $H$ be its proper unfolding (see figure \ref{fig:3x12x2_manifold}). Observe that $M$ is indeed a manifold: it partitions $V(G) \setminus M$ into two sets $R_1$ and $R_2$, both bounded by mutually orthogonal red, green, and blue faces (see figure \ref{fig:3x12x2_manifold}). Furthermore, note that $H$ is obtained from $M$ by cutting along seams between red and green faces, and flattening the figure. It follows that $H$ is a proper unfolding of $G$. We show that $H$ admits a perfect lethal set. 

Consider the initial infection $A$ of $H$ illustrated in figure \ref{fig:3x12x2_unfolded_lethal}. Observe that $A$ infects all vertices of $H$ by lemma \ref{lem:forest}, with the exception of two regions on the far left and far right. However, note that upon refolding, the two cells marked with an ``X" in $H$ represent the same cell in $G$. This is sufficient to infect the remaining regions of $H$, and by corollary \ref{cor:ortho_walls}, $A$ is lethal on $G$. Finally, a simple calculation reveals that $|A|$ matches the surface area bound, and therefore is perfect.

We can extend this construction in the $x$ direction by inserting the repeated structure of six columns; this augmentation percolates by lemma \ref{lem:forest} and agrees with the surface area bound for all larger grids. This completes the proof.
\end{proof}

% (0 mod 6, 3, 2)
\begin{figure}[]
\centering
\includegraphics[width=0.8\textwidth]{figures/7/3x12x2.pdf}
\caption{A perfect percolating set for $(3,12,2)$.}
\label{fig:3x12x2}
\end{figure} 

\begin{figure}[]
\centering
\begin{subfigure}{0.45\textwidth}
	\includegraphics[width=\textwidth]{figures/7/3x12x2_manifold_3d.pdf}
	\caption{A manifold of $G= (3,12,2)$.}
	\label{}
\end{subfigure} \hfill%
\begin{subfigure}{0.45\textwidth}
	\includegraphics[width=\textwidth]{figures/7/3x12x2_manifold.pdf}
	\caption{A proper unfolding of $G$.}
	\label{}
\end{subfigure}
\caption{A proper unfolding of $G= (3,12,2)$. Colored rectangles indicate faces of $G$. Dashed lines indicate that cells appear on different layers. }
\label{fig:3x12x2_manifold}
\end{figure} 

\begin{figure}[]
\centering
\includegraphics[width=0.8\textwidth]{figures/7/3x12x2_unfolded_lethal.pdf}
\caption{A percolating set on the proper unfolding of $G= (3,12,2)$.}
\label{fig:3x12x2_unfolded_lethal}
\end{figure} 

% (3 mod 6, 3, 2)
\begin{con}
All $(a,3,2)$ grids with $a \equiv 3 \pmod 6$ and $a >3$ are perfect. 
\end{con}

\begin{proof}
%(This construction is the same as the previous one, except the the final four columns are augmented slightly to accommodate the $0 \pmod 3$ requirement. Instead of deriving a proper unfolding, it is probably easier to simply show that this small change is sufficient to guarantee lethality, and agrees with the S.A. bound.)

%We proceed by induction. Consider the grid $G=(9,3,2)$, and let $A_t$ be the set of infected vertices at time-step $t$ (see Figure \ref{fig:9x3x2}). We show that $A_0$ is lethal on $G_1$ and $G_2$. Observe that after one time-step, the subgraph $G_1[\overline{A_1}]$ is a forest connected to the border of $G_1$, and so by Proposition \ref{prop:immune_regions}, $A_0$ is lethal on $G_1$. Since $A_0$ is lethal on $G_1$, by Lemma \ref{lem:2_neighbor_levels}, it is sufficient to prove that $A_0$ is lethal on $G_2$ under the 2-neighbor bootstrap process. Figure \ref{fig:9x3x1} illustrates the key steps of this process, starting at $t=1$. Infection spreads down rows delineated by red arrows, ultimately infecting all vertices in $G_2$. We conclude that $A_0$ is lethal on $G$ under the 3-neighbor process.

%Let $A = \{1\} \times [3] \times [2]$, $B = \{8,9\} \times [3] \times [2]$ and $X = \{2,3,4,5,6,7\} \times [3] \times [2]$ be components of $G$ (see Figure \ref{fig:9x3x2}). Denote by $AX^kB$ the graph obtained by inserting $k$ copies of $X$ between components $A$ and $B$. Let $A_0^k$ be a lethal set on $AX^kB$. We show that $A_0^{k+1}$ is lethal on $AX^{k+1}B$. 

Let $G=(6k+3,3,2)$ be a grid such that $k>0$. Let $A = \{1\} \times [3] \times [2]$, $B = \{6k+2, 6k+3\} \times [3] \times [2]$, and $X_i = \{6i+2,6i+3,6i+4,6i+5,6i+6,6i+7\} \times [3] \times [2]$ for $i \in [k - 1] \cup \{0\}$, be components of $G$. Denote by $AX^kB$ the union of components $A \cup X_0 \cup \dots \cup X_{k-1} \cup B$, and note that $G=AX^kB$. Let $A_t^k \subseteq V(G)$ be the set of infectious vertices in $G$ at time $t$, and suppose that each $X_i$ contains the same pattern of infected vertices (see Figure \ref{fig:9x3x2}). We show that $A_0^k$ is lethal and perfect. 

Consider the union of components $AX^k = A \cup X_0 \cup \dots \cup X_{k-1}$ (see Figure \ref{fig:19x3x2}). Let $L_1$ and $L_2$ be the top and bottom levels $AX^k$, respectively. Observe that after one time-step, the subgraph $L_1[\overline{A_1^k}]$ is a forest connected to the border of $L_1$, and so by Proposition \ref{prop:immune_regions}, $A_0^k$ is lethal on $L_1$. 

Now consider $G$ and observe that the top level becomes fully infected (see Figure \ref{fig:21x3x2}). Therefore, by Lemma \ref{lem:2_neighbor_levels}, it is sufficient to prove that $A_0^k$ is lethal on the bottom level under the 2-neighbor bootstrap process. Figure \ref{fig:9x3x1} illustrates the key steps of this process on the smaller grid $AXB$, starting at $t=1$. Infection spreads down rows delineated by red arrows, ultimately infecting all vertices in the bottom level. We conclude that $A_0^k$ is lethal on $G$ under the 3-neighbor process.

To prove that $A_0^k$ is perfect, observe that $|A_0^k| = 3 + 10k + 4$. The surface area bound for $G=(6k+3,3,2)$ is given by
$$\frac{(3)(6k+3) + (3)(2) + (2)(6k+3)}{3} = \frac{30k + 21}{3} = 10k+7.$$
Since these two values are equal, $A_0^k$ is tight and lethal, and therefore perfect.
\end{proof}

\begin{figure}[]
\centering
\includegraphics[width=0.3\textwidth]{figures/7/3x9x2.pdf}
\caption{The components $A$, $X$, $B$ on $G=AXB$ with infectious set $A_0$.}
\label{fig:9x3x2}
\end{figure} 

\begin{figure}[]
\centering
\includegraphics[width=0.8\textwidth]{figures/7/3x19x2.pdf}
\caption{An infection on $AX^3$, $t=0$ and $t=1$.}
\label{fig:19x3x2}
\end{figure} 

\begin{figure}[]
\centering
\includegraphics[width=0.8\textwidth]{figures/7/3x21x2.pdf}
\caption{An infection on $G$.}
\label{fig:21x3x2}
\end{figure} 

\begin{figure}[]
\centering
\includegraphics[width=0.8\textwidth]{figures/7/3x9x1.pdf}
\caption{The 2-neighbor process on $(9,3,1)$ for $t=1$, $t \in \{2,3,4,5,6\}$, and $t \in \{7,8,9,10,11,12,13,14\}$.}
\label{fig:9x3x1}
\end{figure} 

% (2 mod 6, 5 mod 6, 2)

\begin{figure}[]
\centering
\includegraphics[width=0.8\textwidth]{figures/7/11x20x2.pdf}
\caption{A perfect percolating set for $(11,20,2)$.}
\label{fig:11x20x2}
\end{figure} 

\begin{figure}[]
\centering
\begin{subfigure}{0.45\textwidth}
	\includegraphics[width=\textwidth]{figures/7/11x20x2_manifold_3d.pdf}
	\caption{A manifold of $G= (11,20,2)$.}
	\label{}
\end{subfigure} \hfill%
\begin{subfigure}{0.45\textwidth}
	\includegraphics[width=\textwidth]{figures/7/11x20x2_manifold.pdf}
	\caption{A proper unfolding of $G$.}
	\label{}
\end{subfigure}
\caption{A proper unfolding of $G= (11,20,2)$. Colored rectangles indicate faces of $G$. Dashed lines indicate that cells appear on different layers. }
\label{fig:11x20x2_manifold}
\end{figure} 

\begin{figure}[]
\centering
\includegraphics[width=0.8\textwidth]{figures/7/11x20x2_unfolded_lethal.pdf}
\caption{A percolating set on the proper unfolding of $G= (11,20,2)$.}
\label{fig:11x20x2_unfolded_lethal}
\end{figure} 

% a,b, MUST BE A CERTAIN SIZE
\begin{con}
All $(a,b,2)$ grids with $a,b \in \{2,5\} \pmod 6$, $a \neq b \pmod 6$, and $a,b > 2$ are perfect. 
\end{con}

\begin{proof}
Let $G$ be an $(a,b,2)$ grid with $a,b \in \{2,5\} \pmod 6$ and $a \neq b \pmod 6$, and let $M$ be a manifold of $G$ and $H$ be its proper unfolding (figure \ref{fig:11x20x2_manifold}). Note that $M$ partitions the vertices of $V(G) \setminus M$ into two disjoint sets $R_1$ and $R_2$, both bounded by mutually orthogonal red, green, and blue faces. Note, also, that $H$ is obtained from $M$ by cutting along seams between red and green faces, and flattening the figure. Therefore, $H$ is a proper unfolding of $G$. We show that $H$ admits a perfect lethal set. 

Consider the initial infection $A$ of $H$ as shown in figure \ref{fig:11x20x2_unfolded_lethal}. By lemma \ref{lem:forest}, $A$ infects all vertices of $H$, with the exception of two regions on the left- and right-most sides of the grid. However, note that the vertices labeled ``X" in figure \ref{fig:11x20x2_unfolded_lethal} represent that same vertex in $G$. This permits $A$ to infect the remaining healthy vertices, thereby proving that $A$ is lethal on $H$. By corollary \ref{cor:ortho_walls}, $A$ is lethal on $G$. Finally, a simple calculation shows that $|A|$ satisfies the surface area bound on $G$, and so $A$ is perfect. 

The above construction holds for $a \geq 5$ and $b \geq 8$. It can be extended in the $x$ direction by inserting a block of width 6, representing the repeating vertical snaking pattern. Similarly, it can be extended in the $y$ direction by inserting a block of height 6, representing the horizontal snaking pattern. Both such augmentations spawn infections that satisfy the conditions of lemma \ref{lem:forest}, and a simple calculation reveals that they agree with the surface area bound. This concludes the proof. 
\end{proof}

% (3 mod 6, 0 mod 6, 2) 

% a,b MUST BE A CERTAIN SIZE
\begin{con}
All $(a,b,2)$ grids with $a,b \in \{0,3\} \pmod 6$ and $a \neq b \pmod 6$ are perfect. 
\end{con}

\begin{proof}
Consider the $(21,12,2)$ grid $G$ shown in figure \ref{fig:12x21x2}. Let $H$ be a unfolding of $G$ (figure \ref{fig:12x21x2_unfolded}). Observe that $H$ is proper: three mutually orthogonal faces of $G_1$ are shown by blue, green and dashed red regions, and mutually orthogonal faces of $G_2$ are shown by the red, green and dashed blue regions. We show that $H$ admits a lethal set of size $\text{S.A.}(12,21,2) = 106$. Consider such a set, as shown in figure \ref{fig:12x21x2_unfolded_lethal}. (Observe that this is the same set as shown in figure \ref{fig:12x21x2}.) By lemma \ref{lem:forest}, this set percolates with the exception of two $C_4$s in the top and bottom of the grid. However, notice that one of these cells is a duplicate of an already infected cell. (This duplication is a consequence of the proper unfolding of $G$.) Therefore, $H$ admits a lethal set, and by corollary \ref{cor:ortho_walls}, $G$ is perfect.

For all larger grids, observe that the snaking corridor in the left side $G$ can be extended by multiples of 6 in both the $x$ and $y$ directions. These resulting grid still percolates under lemma \ref{lem:forest}. A simple calculation verifies that such an alteration produces initial infections at the surface area bound. 
\end{proof}

\begin{figure}[]
\centering
\includegraphics[width=0.8\textwidth]{figures/7/12x21x2.pdf}
\caption{A perfect percolating set for $(12,21,2)$.}
\label{fig:12x21x2}
\end{figure} 

\begin{figure}[]
\centering
\begin{subfigure}{0.45\textwidth}
	\includegraphics[width=\textwidth]{figures/7/12x21x2_manifold_3d.pdf}
	\caption{A manifold of $G= (12,21,2)$.}
	\label{}
\end{subfigure} \hfill%
\begin{subfigure}{0.45\textwidth}
	\includegraphics[width=\textwidth]{figures/7/12x21x2_unfolded.pdf}
	\caption{A proper unfolding of $G$.}
	\label{}
\end{subfigure}
\caption{A proper unfolding of $G= (12,21,2)$. Colored rectangles indicate faces of $G$. Dashed lines indicate that cells appear on different layers. }
\label{fig:12x21x2_unfolded}
\end{figure} 

\begin{figure}[]
\centering
\includegraphics[width=0.8\textwidth]{figures/7/12x21x2_unfolded_lethal.pdf}
\caption{A percolating set on the proper unfolding of $G= (12,21,2)$.}
\label{fig:12x21x2_unfolded_lethal}
\end{figure} 

\section{Thickness 3}

\begin{con}
All $(a,3,3)$ grids $G$ with $a \equiv 0 \pmod 2$ and $a > 2$ are perfect. 
\end{con}

\begin{proof}
Let $G=(2k,3,3)$ be a grid such that $k > 1$. Let $A = \{1,2,3\} \times [3] \times [3]$, $B = \{2k\} \times [3] \times [3]$, and $X_i = \{2i+2,2i+3\} \times [3] \times [3]$ for $i \in [k-2]$, be components of $G$. Denote by $AX^kB$ the union of components $A \cup X_1 \cup \dots \cup X_{k} \cup B$, and note that $G=AX^kB$. Let $A_t^k \subseteq V(G)$ be the set of infectious vertices in $G$ at time $t$, and suppose that each $X_i$ contains the same pattern of infected vertices (see Figure \ref{fig:3x6x3}). We show that $A_0^k$ is lethal and perfect. 

Consider the union of components $AX^k = A \cup X_1 \cup \dots \cup X_{k}$ (see Figure \ref{fig:3x13x3}). Let $L_1$, $L_2$ and $L_3$ be the top, middle and bottom levels of $AX^k$, respectively. Observe that after one time-step, the subgraph of $L_2 \setminus \{2k-1\} \times [3] \times \{2\}$ induced by $\overline{A_0^k}$ is acyclic with no border-to-border vertices, and so by Proposition \ref{prop:immune_regions}, $A_0^k$ infects all vertices of $L_2$ apart from those in the rightmost column (labeled ``X"; see Figure \ref{fig:3x13x3}). Therefore, by Lemma \ref{lem:2_neighbor_levels}, all vertices in $L_1$ apart from the rightmost column (labeled ``Y") become infected by the 2-neighbor process. Similarly, the red arrow in Figure \ref{fig:3x13x3}) shows the path of infection in $L_3$. 

Consider these observations in the context of $G$. Figure \ref{fig:3x14x3} shows that it takes 7 additional time-steps to fully infect $L_1$ and $L_2$. By Lemma \ref{lem:2_neighbor_levels}, the remaining healthy vertices in $L_3$ become infected. We therefore conclude that $A_0^k$ is lethal on $G$ under the 3-neighbor process.

To prove that $A_0^k$ is perfect, observe that $|A_0^k| = 8 + 4(k-2) + 3=4k+3$. The surface area bound for $G=(2k,3,3)$ is given by
$$\frac{(2k)(3) + (3)(3) + (3)(2k)}{3} = \frac{12k + 9}{3} = 4k+3.$$
Since these two values are equal, $A_0^k$ is tight and lethal, and therefore perfect.
\end{proof}

\begin{figure}[]
\centering
\includegraphics[width=0.2\textwidth]{figures/7/3x6x3.pdf}
\caption{The components $A$, $X$, $B$ on $G=AXB$ with infectious set $A_0$.}
\label{fig:3x6x3}
\end{figure} 

\begin{figure}[]
\centering
\includegraphics[width=0.8\textwidth]{figures/7/3x13x3.pdf}
\caption{An infection on $AX^5$, $t=0$ and $t=1$.}
\label{fig:3x13x3}
\end{figure} 

\begin{figure}[]
\centering
\includegraphics[width=0.5\textwidth]{figures/7/3x14x3_numbered_heatmap.pdf}
\caption{An infection on $G$.}
\label{fig:3x14x3}
\end{figure} 

\begin{con}
All $(a,b,3)$ grids $G$ with $a \equiv 3 \pmod 6$ and $b \equiv 1 \pmod 2$ are perfect. 
\end{con}

\begin{proof}
Consider the grid $H=(a+2,b+2,1)$, and observe that such a grid admits an optimal percolating set by construction \ref{con:snake}. Note that
$$\text{SA}(a,b,3) = \ceil{\text{SA}(a+2,b+2,1)} - 3.$$
We show that an unfolding of $G$ can be obtained from a simple augmentation of $H$. Let $H'$ be the grid obtained by deleting the four vertices in the bottom, right-most corner of $H$ (see figure \ref{fig:17x25x1_unfolded}). Consider the folding pattern illustrated in figure \ref{fig:17x25x1_manifold}, and observe that the pairs of vertices adjacent to the deleted region are duplicates of each other. (In other words, consider folding up the red and green regions in figure \ref{fig:17x25x1_manifold}, and notice that this operation causes vertices to overlap.) Taking this into account, the unfolding of $G$ percolates by lemma \ref{lem:forest}. Since $H$ admits an optimal percolating set of size $\ceil{\text{SA}(a+2,b+2,1)}$, and precisely 3 of the vertices deleted from $H$ to obtain $H'$ were infected, it follows that the unfolding of $G$ percolates at the lower bound. Finally, by lemma \ref{cor:ortho_walls}, since the unfolding of $G$ is proper and percolates at the lower bound, $G$ is perfect.
\end{proof}

\begin{figure}[]
\centering
\includegraphics[width=0.8\textwidth]{figures/7/17x25x1_unfolded.pdf}
\caption{A percolating set on the proper unfolding $H'$ of $G= (15,23,3)$.}
\label{fig:17x25x1_unfolded}
\end{figure}

\begin{figure}[]
\centering
\begin{subfigure}{0.45\textwidth}
	\includegraphics[width=\textwidth]{figures/7/17x25x1_manifold_3d.pdf}
	\caption{A manifold of $G= (15,23,3)$.}
	\label{}
\end{subfigure} \hfill%
\begin{subfigure}{0.45\textwidth}
	\includegraphics[width=\textwidth]{figures/7/17x25x1_manifold.pdf}
	\caption{A proper unfolding of $G$.}
	\label{}
\end{subfigure}
\caption{A proper unfolding of $G= (15,23,3)$. Colored rectangles indicate faces of $G$. }
\label{fig:17x25x1_manifold}
\end{figure} 

\begin{con}
All $(a,3,3)$ grids $G$ with $a \equiv 0 \pmod 2$ are perfect. 
\end{con}

% COMMENT
% COMMENT
% COMMENT
% COMMENT

\begin{comment}

\section{Individual constructions}

In this section, we diagram lethal set constructions for single grids. The initial infection $A$ is colored red, and all other cells are labeled with the time $t$ that they are first infected. 

% (5, 2, 2)
\begin{con}
The grid $(5,2,2)$ is perfect.
\end{con}

\begin{proof}
See figure \ref{fig:2x5x2_numbered_heatmap}.
\end{proof}

\begin{figure}[]
\centering
\includegraphics[width=0.2\textwidth]{figures/7/2x5x2_numbered_heatmap.pdf}
\caption{}
\label{fig:2x5x2_numbered_heatmap}
\end{figure} 

% (5, 5, 5)
\begin{con}
The grid $(5,5,5)$ is perfect.
\end{con}

\begin{proof}
See figure \ref{fig:5x5x5_numbered_heatmap}.
\end{proof}

\begin{figure}[]
\centering
\includegraphics[width=0.2\textwidth]{figures/7/5x5x5_numbered_heatmap.pdf}
\caption{}
\label{fig:5x5x5_numbered_heatmap}
\end{figure}

% (8, 5, 5)
\begin{con}
The grid $(8,5,5)$ is perfect.
\end{con}

\begin{proof}
See figure \ref{fig:5x8x5_numbered_heatmap}.
\end{proof}

\begin{figure}[]
\centering
\includegraphics[width=0.2\textwidth]{figures/7/5x8x5_numbered_heatmap.pdf}
\caption{}
\label{fig:5x8x5_numbered_heatmap}
\end{figure}

% (a, 3, 3)

% (6, 3, 3)

% (6, 3, 2)

\section{Useful lemmas and observations}

We shall see that similar patterns and structures appear with some regularity in optimal sets. These structures always infect entire regions, and it will be helpful to recognize them within larger grids when they appear. 

\begin{lem}
\label{lem:forest}
Let $G = P_a \Osq P_b$ be a graph and $S \subseteq V(G)$ be a subset of the vertices of $G$. Let $G[\overline{S}]$ be the subgraph of $G$ induced by vertices not in $S$. Then $S$ is a lethal set if and only if $G[\overline{S}]$ is cycle-free, and has no paths between any two boundary vertices. 
\end{lem}

\begin{lem}
\label{lem:walls}
Let $G$ be the grid graph $(a,b,c)$ and let $A$ be a set of infected vertices in $G$. Let $H = G[\overline{A}]$ be the subgraph of $G$ induced by uninfected vertices. Let $\{F, F'\} \subseteq V(G)$ be orthogonal faces of $G$. If $H$ does not contain a path between vertices $v \in F, v' \in F'$, then $G$ percolates.
\end{lem}

\begin{proof}
We proceed by induction on $|V(H)| = abc - |A|$. If $|V(H)| = 0$, then all vertices of $G$ are infected and we are done. Suppose $|V(H)| > 0$, and consider a connected component $Y$ of $H$. By hypothesis, $Y$ does not contain a path between any two orthogonal faces of $G$. Therefore, without loss of generality, there exists a face $X = \{a\} \times \{1, \dots b\} \times \{1, \dots, c\}$ such that $V(Y) \cap X = \emptyset$. Take the largest $i$ such that $X_i = \{a\} \times \{1, \dots b\} \times \{1, \dots, c\}$ contains a vertex of $Y$. Repeat this process to obtain maximal faces $Y_j$ and $Z_k$ in the $b$ and $c$ directions, respectively. 

Observe that this construction gives a vertex $v = (i,j,k) \in V(Y)$, and planes $X_{i+1}, Y_{j+1}, Z_{k+1}$ such that $(X_{i+1} \cup Y_{j+1} \cup Z_{k+1}) \cap  V(Y) = \emptyset$. In particular, note that $(i+1,j,k), (i,j+1,k),(i,j,k+1) \in N(v)$. Since these three vertices belong to $A$ and $v \notin A$, $v$ becomes infected. Furthermore, since $|V(H) \setminus \{v\}| < |V(H)|$, the resulting graph percolates by induction. This completes the proof.
\end{proof}

\begin{cor}
\label{cor:three_walls}
Let $G$ be the grid graph $(a,b,c)$. If a set $A$ is lethal on three mutually orthogonal faces of $G$, then $A$ is lethal on $G$.
\end{cor}

\begin{proof}
Let $X, Y, Z$ be three mutually orthogonal faces of $G$. By hypothesis, $X \cup Y \cup Z \subseteq A_t$ for some time $t$. Therefore, the graph $H = G[\overline{A_t}]$ cannot contain a path between vertices on orthogonal faces of $G$. By lemma \ref{lem:walls}, $G$ percolates.
\end{proof}

\begin{cor}
\label{cor:manifold}
Let $G$ be the grid graph $(a,b,c)$ and let $A$ be a set of infected vertices of $G$. Let $\{R_1, \dots, R_k\} $ be a partition of $V(G) \setminus A$ into sub-grids $(a_1,b_1,c_1), \dots, (a_k,b_k,c_k)$ such that each $R_i$ is bounded by a lethal infection on three mutually orthogonal faces. Then $A$ is lethal on $G$.
\end{cor}

\begin{proof}
By hypothesis and corollary \ref{cor:three_walls}, $A$ is lethal on each $R_i$. Since $A \cup R_1 \cup \dots \cup R_k = V(G)$ and each $R_i$ becomes infected, $A$ must be lethal.
\end{proof}

\begin{defn}
\label{def:manifold}
Let a \emph{manifold} $M$ of the grid graph $G = (a,b,c)$ be any set of vertices $A$ satisfying the conditions of corollary \ref{cor:manifold}.
\end{defn}

\begin{defn}
\label{def:proper_unfolding}
Let a \emph{proper unfolding} of $G = (a,b,c)$ be a planar representation of the manifold of $G$. This can be thought of as a special type of folding net of $G$, such that when assembled the resulting structure satisfies the conditions of corollary \ref{cor:manifold}.
\end{defn}

\begin{cor}
\label{cor:ortho_walls}
Let $M$ be the proper unfolding of a manifold of $G = (a,b,c)$, and let $A$ be a lethal set on $M$. Then $A$ is lethal on $G$.
\end{cor}

\begin{proof}
The proof follows directly from definitions \ref{def:manifold} and \ref{def:proper_unfolding}.
\end{proof}

%\begin{cor}
%\label{cor:three_walls}
%Let $G$ be a grid assembled from sub-grids $G_1, \dots, G_k$. Let $H_1, \dots, H_k$ be sets satisfying the conditions of lemma \ref{lem:three_walls} for sub-grids $G_1, \dots, G_k$, respectively. Then $H = H_1 \cup \dots \cup H_k$ is a lethal set in $G$.
%\end{cor}

%\begin{lem}
%\label{lem:unfold}
%Let $G$ be the grid $(a,b,c)$. Let $H$ be a subgraph induced by the vertices of $G$. If the projection of $H$ in the $x$ direction gives the graph $(b,c)$, in the $y$ direction gives $(a,c)$, and in the $z$ direction gives $(a,b)$, and $H$ percolates, then $G$ percolates.
%\end{lem}

%\begin{proof}
%The projection condition guarantees that $G$ has lethal sets on three perpendicular planes. The result follows from lemma \ref{lem:three_walls}.
%\end{proof}

%We refer to the union of mutually orthogonal faces of sub-grids $G_1, \dots, G_k$ as a \emph{manifold} of $G$. Therefore, corollary \ref{cor:three_walls} says that if a set $H$ is lethal on the manifold of $G$, then $H$ is lethal in $G$. Furthermore, we define a \emph{proper unfolding} of $G$ as a planar representation of the manifold of $G$. This can be thought of as a special type of folding net of $G$, such that when assembled the resulting structure satisfies the conditions of corollary \ref{cor:three_walls}. 

%\begin{cor}
%\label{cor:unfold}
%Something about unfolding $H$ and its planar embedding. Imagine $H$ as a set of folded up pieces of paper. Let $\mathcal{N}$ be the family of unfolded nets comprising $H$. To show that $G$ percolates, it is sufficient to show that all nets $N \in \mathcal{N}$ harbor lethal infections.
%\end{cor}

\end{comment}
