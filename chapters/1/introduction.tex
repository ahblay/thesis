\chapter{Introduction}

\section{Introduction and Definitions}
Let the ordered tuple $(a,b,c)$ represent the $a \times b \times c$ grid $G$ where $a \geq b \geq c$. We refer to $c$ as the ``thickness" of $G$. For example, the tuple $(5,3,3)$ represents a $5 \times 3 \times 3$ grid of thickness 3. We refer to a tuple as ``divisible", or a ``divisibility case", if and only if $ab+bc+ca \equiv 0 \pmod 3$. Observe that the divisibility cases are precisely those grids with integral lower bounds. The divisibility cases of thicknesses belonging to the three residue classes modulo 3 are illustrated in \{Figure something\}.

In the following lemmas, we use the notation $(a,b,c)+(x,y,z) = (a+x, b+y, c+z)$ to represent respective increases of $x$, $y$, and $z$ to the side lengths $a$, $b$, and $c$ of $G$. We note the following: 
\begin{rem}
By applying the recursion, $(a,b,c)+(x,y,z)$ percolates at the lower bound when either:
\begin{enumerate}
\item $(a,b,c), (a,y,z), (x,b,z), (x,y,c)$ all percolate at the lower bound, or;
\item $(x,y,z), (x,b,c), (a,y,c), (a,b,z)$ all percolate at the lower bound.
\end{enumerate}
\end{rem}

We shall call a thickness ``complete" if it can be shown that all divisibility cases in that thickness percolate at the lower bound. In this section, we demonstrate that thickness 5, thickness 6 and thickness 7 are all complete. As these belong to the residue classes 2, 0, and 1 modulo 3, respectively, we then use a recursive construction to show that all larger grids are also complete. \cite{dove}

\subsection{Intuition and Statement}

\subsection{Proof}