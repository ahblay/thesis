\chapter{A Recursive Technique}

In this chapter, we shall present a technique for constructing large \emph{perfect} grids from smaller perfect grids. 

\section{A Helpful Lemma}

Note that there are certain broad structures in a cube that, if present, immediately guarantee it become fully infected. Of greatest importance here is the observation that certain configurations of fully infected sub-cubes (which we shall call blocks) will cause the larger brick to become infected. 

% Describe the structure of these sub-cubes, and prove that they will infect the entire larger brick 

Furthermore, note that if each of these smaller blocks is infected with a minimum lethal set, the composite larger brick will also be infected with a minimum lethal set (barring some considerations for divisibility).

% Handle the divisibility cases, and note that certain augmentations to the recursion allow us to obtain non-divisible optimal grids, if we are cautious with regard to the pieces we use.

\section{Examples and Notation}

% Discuss that the recursion works by assembling any set of compatible n^2 blocks, although it is very rare that it is necessary to use more than 4. 
% Talk about the component pieces necessary to assemble a brick of size (a,b,c).

% Should we talk about the broad behavior of percolating sets?
\section{Regional vs. Temporal Infections}

